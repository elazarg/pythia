\documentclass[a4paper]{article}

\usepackage[english]{babel}
\usepackage[utf8]{inputenc}
\usepackage{amsmath}
\usepackage{graphicx}
\usepackage[colorinlistoftodos]{todonotes}

\usepackage{listings}

\title{Pythia: CPython Bytecode Analysis}

\author{Elazar Gershuni}

\date{\today}

\begin{document}
\maketitle

\begin{abstract}

Like many systems targerting ``managed'' languages, CPython translates Python code to stack-oriented bytecode. This code is relatively dense, with small average instruction size and (mostly) simple execution model. However, such code does not lend itself easily to static analysis. This is mainly due to high reuse of memory locations, and the fact that ``variable names'' --- stack depth at a given point --- is implicit and depends on previous manipulations of the stack pointer.

Pythia is a ``decompiler'' translating CPython stack-oriented code into a more explicit three-address code, arranged in a control flow graph (CFG). 

The expected users of Pythia are people writing code analysis tools for Python.

\end{abstract}

\section{Translation Steps}

The translation consists of 4 steps:
\begin{enumerate}
\item Translating bytecode into BCode, fixing minor issues
\item Translating the sequence of BCode instruction into a CFG graph of basic blocks
\item Generating a CFG graph of an simpler-to-analyze Three-Address-Code (TAC) instructions, grouped into basic blocks
\item Basic optimizations such as live variable analysis and copy propagation
\end{enumerate}

\subsection{Bytecode Instruction}
CPython compiles code into bytecode instructions, and has library support (the \texttt{dis} module) for reading and manipulating these instructions.

The class representing a bytecode instruction is a \texttt{namedtuple} called \texttt{Instruction}.

\subsection{Translating \texttt{Insruction} into \texttt{BCode}}

As a first step, we create an adapter --- a new set of instructions, very similar to the original one, but with a controllable interface.

This step is mostly straight forward, but two problems are needed to be addressed. First, we will need to calculate the effect of an instruction on the depth of the stack, in order to assign a name to each stack-object. The library has a function that calculates the \textit{maximal} effect, but we need to be exact; we simply fix it on a case-by-case basis.

Second, the \texttt{Instruction} \texttt{BREAK\_LOOP} is problematic; it does not contain the target of the break, and both stack effect and target depends on whether it is inside a \texttt{FOR} loop or inside a \texttt{WHILE} loop. Luckily, these loops have different instructions that encode them, so this problem may be handled statically.

\subsection{Building a \texttt{BCode} CFG}

Given a sequence of \texttt{BCode} instructions, we build a control-flow graph. Of course, this graph is already encoded in the instructions, and in a pretty direct manner (after fixing \texttt{BREAK\_LOOP}), so we only make it explicit and standard, using the well-known \texttt{networkx} graph library.

\subsection{Computing Global Stack-depth}

Using this standard graph representation, we now make the most important computation: calculating the ``global'' stack-depth at each point. This is done by running Dijkstra's algorithm over the underlying graph of the CFG, using the stack-effect as a weight.

To elaborate, the total stack-effect on a given path from the entry point to some instruction $t$ defines the depth of the stack on which $t$ will operate. We take the underlying graph, since we want to give consistent variable names to local variables even on unreachable points in the code.

It should be note here that this part of the analysis is only possible when the stack-effect of an instruction is statically computable. The analysis is sound only under the assumption that the stack-depth does not depend on the path.\footnote{Hence Dijkstra seems like an overkill: minimality is both trivial and unimportant. Any path-finding algorithm will give the same results.} To the best of our knowledge, this is the case when the bytecode was compiled from Python source using the CPython compiler. However, we are not aware of a verification phase of this condition in the CPython runtime, so a code that validates this assumption might be executable. (This is not the case e.g. in the Java system, in which the independence of path is an enforced requirement of the runtime type system, and is validated at load time). However, we did not find an example of such violation.

This step is the key to giving local variables names, which makes the difference between the (un)analyzability of stack-oriented code to that of an intermediate representation. The whole operation consists of 3 calls to general graph-library functions.

\subsection{Translating \texttt{BCode} into Three-Address Code}

Knowing the depth at each program point, we consistently give stack locations a name --- $@v1$, $@v2$ and so on --- and translate each instruction to an equivalent command of a home-brewed 3-address code (TAC), described roughly in the following list\footnote{Not all the instructions are actively supported: we do not analyze YIELD, IMPORT or RAISE yet.}:
\begin{lstlisting}
    NOP
    ASSIGN vars = a
    IMPORT var
    BINARY var = a op b
    INPLACE var = op b
    CALL var = func(a1, ..., an)
    IF var JUMP dst 
    FOR var IN var
    RET val
    RAISE val
    YIELD val GIVING dst
    DEL var1, ..., varn
\end{lstlisting}

Having translated each bytecode instruction on the CFG, we now have immediately have a CFG of TAC instructions.

\subsection{Simple Analysis of Three-Address Code}
The TAC representation is suitable for sophisticated analysis, but is very noisy: there are many local variables and many unnecessary reads and writes. We therefore apply two simple analyses / optimizations: constant/copy propagation, and liveness. This allows us to compress the code, transforming multiple instructions into a single one.

The instructions actually contain liveness, kill/gen information, so it is possible to eliminate many \texttt{DEL} instructions.

Note that while performing constant/copy propagation, we do not remove dead writes to source variables (as opposed to stack-machine-originated local ones). It is important since these write are externally visible through calls to the functions \texttt{locals()} or \texttt{globals}. For similar reason we do not propagate source variables, and we do not propagate constants across source variables since we still want the code to be similar to the original source.

\section{What We Support}
Since Pythia works with bytecode, it does not restrict directly the Python constructs used. However, it supports only a (large) subset of the instruction-set of CPython3.5, and certain language constructs will always translate to specific bytecode operations.

Pythia does not support exception handling mechanisms (except, finally, with); YIELD or ASYNC operations; and currently does not support function and class definition, although it should be easy to add.

It should also be relatively straightforward to add support to exception handling, but the amount of code rewrite will be significant since the current stack model is an (unnecessary) simplification of the actual model.
\newpage
\twocolumn
\lstset{language=Python,caption={Python code}}
\begin{lstlisting}
def simple():
    while z:
        x = 1
        y = 2
        z = x + y
\end{lstlisting}
\lstset{language=Python,caption={Three-Address Code,\\ simplified}}
\begin{lstlisting}
0 :
	 NOP
3 :
	 @v0 = not(z)
	 IF @v0 GOTO 34
9 :
	 x = 1
	 y = 2
	 @v0 = x + y
	 z = @v0
	 IF True GOTO 3
34 :
	 NOP
35 :
	 RETURN None
\end{lstlisting}
\newpage
\lstset{caption={Three-Address Code,\\ not simplified}}
\begin{lstlisting}
0 :	 NOP
3 :	 DEL ('@v0',)
3 :	 @v0 = z
3 :	 @v0 = not(@v0)
3 :	 IF @v0 GOTO 34
3 :	 DEL ('@v0',)
9 :	 DEL ('@v0',)
9 :	 @v0 = 1
9 :	 DEL ('x',)
9 :	 x = @v0
9 :	 DEL ('@v0',)
9 :	 DEL ('@v0',)
9 :	 @v0 = 2
9 :	 DEL ('y',)
9 :	 y = @v0
9 :	 DEL ('@v0',)
9 :	 DEL ('@v0',)
9 :	 @v0 = x
9 :	 DEL ('@v1',)
9 :	 @v1 = y
9 :	 @v0 = @v0 + @v1
9 :	 DEL ('@v1',)
9 :	 DEL ('z',)
9 :	 z = @v0
9 :	 DEL ('@v0',)
9 :	 IF True GOTO 3
34 :	 NOP
35 :	 DEL ('@v0',)
35 :	 @v0 = None
35 :	 RETURN @v0
35 :	 DEL ('@v0',)
\end{lstlisting}

\end{document}
